\documentclass[12pt,a4paper,oneside]{report}
\usepackage[latin1]{inputenc}
\usepackage{amsmath}
\usepackage{amsfonts}
\usepackage{amssymb}
\usepackage{graphicx}
\usepackage{listings}
\usepackage{xcolor}
\colorlet{mygray}{black!30}
\colorlet{mygreen}{green!60!blue}
\colorlet{mymauve}{red!60!blue}
\lstset{
	backgroundcolor=\color{gray!10},  
	basicstyle=\ttfamily,
	columns=fullflexible,
	breakatwhitespace=false,      
	breaklines=true,                
	captionpos=b,                    
	commentstyle=\color{mygreen}, 
	extendedchars=true,              
	frame=single,                   
	keepspaces=true,             
	keywordstyle=\color{blue},      
	language=c,                 
	numbers=none,                
	numbersep=5pt,                   
	numberstyle=\tiny\color{blue}, 
	rulecolor=\color{mygray},        
	showspaces=false,               
	showtabs=false,                 
	stepnumber=5,                  
	stringstyle=\color{mymauve},    
	tabsize=3,                      
	title=\lstname                
}
\usepackage[width=0.00cm, height=0.00cm, left=2.50cm, right=2.50cm, top=2.50cm, bottom=2.50cm]{geometry}


\begin{document}
\begin{titlepage}
	\newcommand{\HRule}{\rule{\linewidth}{0.5mm}} % Defines a new command for the horizontal lines, change thickness here
	\center % Center everything on the page
	%-----------------------------------------------------------------------
	%	LOGO SECTION
	%-----------------------------------------------------------------------
	
	\includegraphics[scale = 0.9]{images/logo.png}\\[2cm] % Include a department/university logo - this will require the graphicx package
	
	%-----------------------------------------------------------------------
	%	HEADING SECTIONS
	%-----------------------------------------------------------------------
	\text{\LARGE Fraunhofer IPA}\\[1.5cm] % Name of your university/college
	\textsc{\Large Robotics and Assistance Systems}\\[0.5cm] % Major heading such as course name
	%\textsc{\large Assignment 1}\\[0.5cm] % Minor heading such as course title
	
	%-----------------------------------------------------------------------
	%	TITLE SECTION
	%-----------------------------------------------------------------------
	
	\HRule \\[0.4cm]
	{ \huge \bfseries Porting ROS 1 packages to ROS 2}\\[0.4cm] % Title of your document
	\HRule \\[1.5cm]
	
	%-----------------------------------------------------------------------
	%	AUTHOR SECTION
	%-----------------------------------------------------------------------
	\vspace{25mm}
%	\begin{minipage}{0.9\textwidth}
%		\begin{flushleft} \large
%			
%			\emph{Authors:}\\
%			Ragesh \text{Ramachandran}\\
%			Harsh \text{Deshpande}
%			
%		\end{flushleft}
%	\end{minipage}
%	~

	%----------------------------------------------------------------------------------------
	%	DATE SECTION
	%----------------------------------------------------------------------
	\vspace{15mm}	
	{\large \today}\\[4cm]
	%-----------------------------------------------------------------------
	\vfill % Fill the rest of the page with whitespace
	
\end{titlepage}

\tableofcontents
\newpage
\chapter{Installing ROS 2}

\section{Set Locale}
Make sure to set a locale that supports UTF-8. If you are in a minimal environment such as a Docker container, the locale may be set to something minimal like POSIX.
\begin{lstlisting}
sudo locale-gen en_US en_US.UTF-8
sudo update-locale LC_ALL=en_US.UTF-8 LANG=en_US.UTF-8
export LANG=en_US.UTF-8
\end{lstlisting}

\section{Add the ROS2 apt repository}
To install the Debian packages you will need to add our Debian repository to your apt sources. First you will need to authorize our gpg key with apt like this
\begin{lstlisting}
sudo apt update && sudo apt install curl gnupg2 lsb-release
curl http://repo.ros2.org/repos.key | sudo apt-key add -
\end{lstlisting}
And then add the repository to your sources list:
\begin{lstlisting}
sudo sh -c 'echo "deb [arch=amd64,arm64] http://packages.ros.org/ros2/ubuntu `lsb_release -cs` main" > /etc/apt/sources.list.d/ros2-latest.list'
\end{lstlisting}

\section{Install development tools and ROS tools}
\begin{lstlisting}
sudo apt update && sudo apt install -y \
build-essential \
cmake \
git \
python3-colcon-common-extensions \
python3-lark-parser \
python3-pip \
python-rosdep \
python3-vcstool \
wget
# install some pip packages needed for testing
python3 -m pip install -U \
argcomplete \
flake8 \
flake8-blind-except \
flake8-builtins \
flake8-class-newline \
flake8-comprehensions \
flake8-deprecated \
flake8-docstrings \
flake8-import-order \
flake8-quotes \
pytest-repeat \
pytest-rerunfailures \
pytest \
pytest-cov \
pytest-runner \
setuptools
# install Fast-RTPS dependencies
sudo apt install --no-install-recommends -y \
libasio-dev \
libtinyxml2-dev
\end{lstlisting}

\section{Install dependencies using rosdep}
\begin{lstlisting}
sudo rosdep init
rosdep update

rosdep install --from-paths src --ignore-src --rosdistro crystal -y --skip-keys "console_bridge fastcdr fastrtps libopensplice67 libopensplice69 python3-lark-parser rti-connext-dds-5.3.1 urdfdom_headers"
python3 -m pip install -U lark-parser
\end{lstlisting}

\section{Building the code in the Workspace}
\begin{lstlisting}
cd ~/ros2_ws/
colcon build --symlink-install --packages-ignore qt_gui_cpp rqt_gui_cpp
\end{lstlisting}
\chapter{NodeHandle to NodeClass}

\section{NodeHandle}
A NodeHandle is an object which represents a ROS node. It provides startup and shutdown of the internal node inside a roscpp program and it provides an extra layer of namespace resolution that can make writing subcomponents easier.\\
When the first \textit{ros::NodeHandle} is created it will call \textit{ros::start()}, and when the last \textit{ros::NodeHandle} is destroyed, it will call \textit{ros::shutdown()}.
\begin{lstlisting}
ros::NodeHandle nh;
\end{lstlisting}

\section{Node Class}
\begin{lstlisting}
ros::NodeHandle nh;
\end{lstlisting}



\chapter{Publisher Subscriber}

\chapter{ROS Bridge}

\chapter{ROS spin}
\end{document}